% ==============================================================================
% TCC - Matheus de Oliveira Lima
% Capítulo 1 - Introdução
% ==============================================================================
\chapter{Introdução}
\label{sec-intro}

%\hl{Texto.}

%\hrulefill

%A \textbf{Introdução} deve conter de \textbf{3 a 5 páginas}. Primeiramente, deve ser colocada a descrição do trabalho, a qual apresenta o contexto do trabalho e a definição do escopo do mesmo. Deve-se delimitar o escopo do trabalho de forma que haja condições técnicas suficientes para que o mesmo seja concluído em tempo hábil.

O Sistema Marvin surgiu como uma iniciativa para ajudar a gestão acadêmica e administrativa da Universidade Federal do Espírito Santo (UFES), complementando os sistemas institucionais existentes. Desenvolvido originalmente em 2019 como um projeto de extensão, o Marvin oferece suporte a demandas específicas de departamentos cursos e programas de pós-graduação - como acompanhamento de egressos.

Este trabalho tem como foco melhorar a interface do Sistema Marvin que foi criado para ajudar na gestão administrativa da faculdade, diferenciando o padrão visual de acordo com os diferentes papéis de usuários (como professores, coordenadores, alunos, entre outros).

%%% Início de seção. %%%
\section{Motivação e Justificativa}
\label{sec-intro-motjus}

%A \textbf{Motivação} apresenta as circunstancias que interferiram na escolha do tema. A \textbf{Justificativa} apresenta o porquê da escolha do tema, o problema a ser resolvido e a relevância do trabalho, referindo-se a estudos anteriores sobre o tema, ressaltando suas eventuais limitações e destacando a necessidade de se continuar pesquisando o assunto.

A motivação para este trabalho surge da necessidade de atender demandas específicas de departamentos e programas de pós-graduação...

A justificativa para esse trabalho está na necessidade de melhorar a experiência dos usuários do Sistema Marvin. Atualmente, ele funciona com páginas genéricas baseadas em operação CRUD, sem levar em conta o papel específico que cada usuário exerce dentro da instituição. Isso acaba tornando o uso do sistema mais confuso e pouco eficiente, especialmente para quem não tem familiaridade com o funcionamento interno das atividades administrativas e acadêmicas.

Desenvolver interfaces personalizadas, que se adaptem ao perfil de cada usuário, pode aumentar a usabilidade do sistema e agilizar tarefas rotineiras. Os trabalhos anteriores deram mais ênfase e esforço nas funcionalidades dentro do sistema administrativo, mas também é importante a experiência do usuário e a adaptação do sistema ao contexto real de uso. Por isso, é relevante continuar pesquisando e desenvolvendo soluções que não apenas funcionem tecnicamente, mas que também sejam intuitivas e úteis no dia a dia dos usuários.

%%% Início de seção. %%%
\section{Objetivos}
\label{sec-intro-obj}

%Nesta subseção, deve ser descrito o \textbf{objetivo geral} do trabalho, detalhando em seguida, seus \textbf{objetivos específicos}.

%O \textbf{Objetivo Geral} expressa a finalidade do trabalho: para quê? Deve ter coerência direta com o tema do trabalho e ser apresentado em uma frase que inicie com um verbo no infinitivo. O objetivo geral do trabalho está relacionado ao resultado principal do trabalho.

%%% Objetivo Geral %%%
O \textbf{objetivo geral} é desenvolver interfaces personalizadas no sistema Marvin, adaptando as páginas de acordo com o papel (Role) de cada usuário, com o objetivo de tornar a navegação mais intuitiva e funcional para diferentes perfis dentro da universidade.

%Os \textbf{Objetivos Específicos} apresentam os detalhes e/ou desdobramentos do objetivo geral que levam a resultados intermediários e relevantes para alcançar o objetivo geral. Sempre será mais de um objetivo específico, todos iniciando com verbo no infinitivo.

%OsObjetivos Específicos apresentam os detalhes e/ou desdobramento do objetivogeral. Sempre serão mais de um objetivo, todos iniciando com verbo no infinitivo queapresente tarefas parciais de pesquisa em prol da execução do objetivo geral.


%%% Início de seção. %%%
\section{Método de Desenvolvimento do Trabalho}
\label{sec-intro-met}

%Nesta subseção, deve ser apresentado o \textbf{Método de Desenvolvimento} (ou o \textbf{Método de Pesquisa}, quando for o caso) do trabalho. Aqui são apresentados os procedimentos/técnicas que serão usados durante o desenvolvimento do trabalho. 

O desenvolvimento deste trabalho será feito em etapas, seguindo uma abordagem prática baseada no processo de engenharia de software, com foco em desenvolvimento web. A metodologia utilizada envolve análise dos requisitos, análise dos papéis dos usuários, planejamento das interfaces, implementação e validação.

Inicialmente, será feita uma análise do sistema Marvin, com foco no subsistema administrativo, para entender como os diferentes tipos de usuários (como administradores, gestores de departamento, coordenadores de curso e programas de pós-graduação) interagem com as funcionalidades existentes. A partir disso, serão identificadas as principais funções associadas a cada papel e os pontos em que a interface atual não atende bem às necessidades específicas desses perfis.

Com base nessa análise, serão projetadas novas interfaces personalizadas, utilizando ferramentas modernas de desenvolvimento frontend. A implementação será feita por meio de tecnologias web já utilizadas no projeto Marvin, respeitando a estrutura já existente e buscando garantir compatibilidade com o sistema atual.

Por fim, será realizada uma etapa de testes com usuários reais ou simulações de uso, para avaliar a usabilidade das novas interfaces. A documentação do processo completo, incluindo decisões técnicas e dificuldades encontradas, também fará parte do trabalho final.

%%% Início de seção. %%%
\section{Cronograma}
\label{sec-intro-crono}

O cronograma de execução apresenta a distribuição no tempo das atividades que serão realizadas ao longo do desenvolvimento do trabalho. As atividades foram definidas com base nos objetivos específicos apresentados na Subseção~\ref{sec-intro-obj}, garantindo que cada uma delas contribua diretamente para a obtenção do objetivo geral.

\begin{itemize}
	\item \textbf{Atividade 1: Levantamento de requisitos e análise de papéis do sistema} \\
	Analisar o sistema Marvin, especialmente o subsistema administrativo, e identificar os diferentes tipos de usuários e suas respectivas funções.
	
	\item \textbf{Atividade 2: Avaliação da usabilidade atual} \\
	Estudar as interfaces existentes no sistema e levantar as principais dificuldades enfrentadas por cada perfil de usuário.
	
	\item \textbf{Atividade 3: Projeto das interfaces personalizadas} \\
	Definir a estrutura, layout e fluxo de navegação das novas interfaces específicas para cada Role.
	
	\item \textbf{Atividade 4: Implementação das interfaces - Parte 1} \\
	Desenvolver e integrar as interfaces personalizadas de 1 ou 2 papéis principais de usuário, como Administrador e Gestor de Curso.
	
	\item \textbf{Atividade 5: Implementação das interfaces - Parte 2} \\
	Finalizar o desenvolvimento das interfaces dos demais papéis e garantir a integração com o sistema existente.
	
	\item \textbf{Atividade 6: Testes, validação e documentação} \\
	Realizar testes com usuários (ou simulações), ajustar problemas encontrados e documentar todo o processo de desenvolvimento.
\end{itemize}

\begin{table}[h]
	\centering
	\caption{Cronograma de execução das atividades}
	\begin{tabular}{l|c|c|c|c|c|c|}
		\cline{2-7}
		\multicolumn{1}{c|}{} & \textbf{Mês 1} & \textbf{Mês 2} & \textbf{Mês 3} & \textbf{Mês 4} & \textbf{Mês 5} & \textbf{Mês 6} \\ \hline
		\multicolumn{1}{|l|}{\textbf{Atividade 1}} & X &   &   &   &   &   \\ \hline
		\multicolumn{1}{|l|}{\textbf{Atividade 2}} &   & X &   &   &   &   \\ \hline
		\multicolumn{1}{|l|}{\textbf{Atividade 3}} &   &   & X &   &   &   \\ \hline
		\multicolumn{1}{|l|}{\textbf{Atividade 4}} &   &   & X & X &   &   \\ \hline
		\multicolumn{1}{|l|}{\textbf{Atividade 5}} &   &   &   & X & X &   \\ \hline
		\multicolumn{1}{|l|}{\textbf{Atividade 6}} &   &   &   &   &   & X \\ \hline
	\end{tabular}
\end{table}