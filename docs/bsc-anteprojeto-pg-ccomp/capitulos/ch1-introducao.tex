% ==============================================================================
% TCC - Matheus de Oliveira Lima
% Capítulo 1 - Introdução
% ==============================================================================
\chapter{Introdução}
\label{sec-intro}

%\hl{Texto.}

%\hrulefill

%A \textbf{Introdução} deve conter de \textbf{3 a 5 páginas}. Primeiramente, deve ser colocada a descrição do trabalho, a qual apresenta o contexto do trabalho e a definição do escopo do mesmo. Deve-se delimitar o escopo do trabalho de forma que haja condições técnicas suficientes para que o mesmo seja concluído em tempo hábil.

Ao observar os períodos de entrada de estudantes (início de cada semestre estudantil), fazer os processos de alocação de professores a cargos internos, colocar estudantes de Graduação e Pós-Graduação em disciplinas, entre outras tarefas administrativas, se torna um processo muito trabalhoso manualmente; Com intuito de facilitar esse isso está sendo desenvolvido o Marvin, para gerenciar tarefas de ensino e pesquisa dentro da universidade.

Este trabalho tem como foco melhorar a interface do Sistema Marvin, que atualmente funciona com CRUD's padrões e pouco adaptados aos diferentes tipos de usuários. Esse sistema foi criado para ajudar na gestão de tarefas de ensino e pesquisa dentro da universidade, mas hoje apresenta páginas padronizadas, sem diferenciação para os distintos papéis de usuários (como professores, coordenadores, alunos, entre outros).


%%% Início de seção. %%%
\section{Motivação e Justificativa}
\label{sec-intro-motjus}

%A \textbf{Motivação} apresenta as circunstancias que interferiram na escolha do tema. A \textbf{Justificativa} apresenta o porquê da escolha do tema, o problema a ser resolvido e a relevância do trabalho, referindo-se a estudos anteriores sobre o tema, ressaltando suas eventuais limitações e destacando a necessidade de se continuar pesquisando o assunto.

A motivação para a escolha deste tema foi o interesse pessoal em atuar na área de desenvolvimento web. Desde quando as disciplinas começaram a ficar mais específicas eu começei a me identificar mais com desenvolvimento web, e essa afinidade  influenciou diretamente na decisão de buscar um projeto que permitisse aplicar na prática os conhecimentos adiquiridos ao longo da graduação. Além disso, ter a oportunidade de contribuir para uma aplicação que tem um potencial alto de ser usado dentro da universidade trouxe uma motivação a mais para participar.

A justificativa para esse trabalho está na necessidade de melhorar a experiência dos usuários do Sistema Marvin. Atualmente, ele funciona com páginas genéricas baseadas em operação CRUD, sem levar em conta o papel específico que cada usuário exerce dentro da instituição. Isso acaba tornando o uso do sistema mais confuso e pouco eficiente, especialmente para quem não tem familiaridade com o funcionamento interno das atividades administrativas e acadêmicas.

Desenvolver interfaces personalizadas, que se adaptem ao perfil de cada usuário, pode aumentar a usabilidade do sistema e agilizar tarefas rotineiras. Os trabalhos anteriores deram mais ênfase e esforço nas funcionalidades dentro do sistema administrativo, mas também é importante a experiência do usuário e a adaptação do sistema ao contexto real de uso. Por isso, é relevante continuar pesquisando e desenvolvendo soluções que não apenas funcionem tecnicamente, mas que também sejam intuitivas e úteis no dia a dia dos usuários.

%%% Início de seção. %%%
\section{Objetivos}
\label{sec-intro-obj}

%Nesta subseção, deve ser descrito o \textbf{objetivo geral} do trabalho, detalhando em seguida, seus \textbf{objetivos específicos}.

%O \textbf{Objetivo Geral} expressa a finalidade do trabalho: para quê? Deve ter coerência direta com o tema do trabalho e ser apresentado em uma frase que inicie com um verbo no infinitivo. O objetivo geral do trabalho está relacionado ao resultado principal do trabalho.

%%% Objetivo Geral %%%
O \textbf{objetivo geral} é desenvolver interfaces personalizadas no sistema Marvin, adaptando as páginas de acordo com o papel (Role) de cada usuário, com o objetivo de tornar a navegação mais intuitiva e funcional para diferentes perfis dentro da universidade.

%Os \textbf{Objetivos Específicos} apresentam os detalhes e/ou desdobramentos do objetivo geral que levam a resultados intermediários e relevantes para alcançar o objetivo geral. Sempre será mais de um objetivo específico, todos iniciando com verbo no infinitivo.

%%% Objetivos Específicos %%%
Com base no objetivo geral pode-se desdobrar alguns resultados intermediários como:
\begin{itemize}
	\item Levantar os diferentes tipos de usuários que utilizam o sistema Marvin e mapear as funções associadas a cada papel.
	\item Identificar os pontos de dificuldade na navegação atual do sistema para cada perfil de usuário.
	\item Projetar novas interfaces personalizadas que atendam melhor às necessidades específicas de cada Role.
	\item Implementar as interfaces no sistema, utilizando boas práticas de desenvolvimento web.
	\item Testar as novas interfaces com usuários reais para validar a usabilidade e a eficiência das mudanças realizadas.
	\item Documentar todo o processo de desenvolvimento e as decisões técnicas tomadas ao longo do trabalho.
\end{itemize}


%OsObjetivos Específicos apresentam os detalhes e/ou desdobramento do objetivogeral. Sempre serão mais de um objetivo, todos iniciando com verbo no infinitivo queapresente tarefas parciais de pesquisa em prol da execução do objetivo geral.


%%% Início de seção. %%%
\section{Método de Desenvolvimento do Trabalho}
\label{sec-intro-met}

Nesta subseção, deve ser apresentado o \textbf{Método de Desenvolvimento} (ou o \textbf{Método de Pesquisa}, quando for o caso) do trabalho. Aqui são apresentados os procedimentos/técnicas que serão usados durante o desenvolvimento do trabalho. 

%%% Início de seção. %%%
\section{Cronograma}
\label{sec-intro-crono}

O \textbf{Cronograma de Execução} apresenta a distribuição no tempo das atividades
que deverão ser desenvolvidas ao longo do trabalho. As atividades apresentadas no cronograma devem estar alinhadas com os objetivos apresentados na Subseção~\ref{sec-intro-obj}, ou seja, elas devem ser capazes de produzir os resultados necessários para alcançar os objetivos estabelecidos. Deve-se apresentar uma listagem com a descrição de cada uma dessas atividades e em seguida mostrada uma tabela contendo todas as atividades previstas juntamente com a  previsão do período de execução de cada uma.

Um exemplo seria:
\begin{itemize}
\item Atividade 1: detalhamento ...
\item Atividade 2: detalhamento ...
\item ...
\end{itemize}

\begin{table}[h]
\centering
\begin{tabular}{c|c|c|c|c|c|c|}
\cline{2-7}
\multicolumn{1}{l|}{} & \textbf{(mês)} & \textbf{(mês)} & \textbf{(mês)} & \textbf{(mês)} & \textbf{(mês)} & \textbf{(mês)} \\ \hline
\multicolumn{1}{|c|}{\textbf{Atividade 1}} & X & X &  &  &  &  \\ \hline
\multicolumn{1}{|c|}{\textbf{Atividade 2}} &  & X & X &  &  &  \\ \hline
\multicolumn{1}{|c|}{\textbf{Atividade 3}} &  &  &  & X &  &  \\ \hline
\multicolumn{1}{|c|}{\textbf{Atividade 4}} &  & X &  & X &  &  \\ \hline
\multicolumn{1}{|c|}{\textbf{. . .}} &  &  &  &  & X & X \\ \hline
\end{tabular}
\end{table}