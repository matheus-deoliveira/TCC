% ==============================================================================
% TCC - Matheus de Oliveira Lima
% Capítulo 2 - Fundamentação Teórica e Tecnologias Utilizadas
% ==============================================================================


\chapter{Fundamentação Teórica e Tecnologias Utilizadas}
\label{sec-fundteo}

A fundamentação teórica deste trabalho aborda conceitos essenciais para o desenvolvimento de interfaces personalizadas em sistemas web, com foco na experiência do usuário (UX) e na adaptação de funcionalidades conforme o papel de cada usuário. Além disso, são apresentadas as principais tecnologias utilizadas no desenvolvimento do Sistema Marvin.

\section{Experiência do Usuário (UX) e Usabilidade}

A experiência do usuário (UX) refere-se à qualidade da interação do usuário com o sistema, englobando aspectos como facilidade de uso, eficiência e satisfação~\cite{norman2013design}. Interfaces bem projetadas aumentam a produtividade e reduzem erros, especialmente em sistemas complexos como os utilizados em ambientes acadêmicos.

Segundo Nielsen~\cite{nielsen1994usability}, a usabilidade pode ser dividida em cinco atributos: facilidade de aprendizado, eficiência de uso, facilidade de memorização, taxa de erros e satisfação subjetiva. No contexto do Sistema Marvin, a personalização das interfaces visa melhorar esses atributos para diferentes perfis de usuários.

\section{Personalização de Interfaces e Controle de Acesso por Papéis (Role-Based Access Control)}

A personalização de interfaces consiste em adaptar a apresentação e as funcionalidades do sistema conforme o perfil do usuário. O controle de acesso baseado em papéis (RBAC) é uma abordagem comum para gerenciar permissões em sistemas multiusuário~\cite{sandhu1996role}. Cada papel (por exemplo, administrador, gestor, aluno) possui um conjunto específico de permissões e visualizações, tornando o sistema mais seguro e intuitivo.

\section{Engenharia de Software}
\label{sec-fundteo-engsoft}

A Engenharia de Software é uma área da computação dedicada à especificação, desenvolvimento, manutenção e gerenciamento de sistemas de software de forma sistemática, disciplinada e quantificável~\cite{pressman2016engenharia}. Seu objetivo é garantir a qualidade, confiabilidade e eficiência dos sistemas desenvolvidos, utilizando processos, métodos e ferramentas apropriadas. Entre as principais atividades da engenharia de software estão a definição de requisitos, o projeto de software, a implementação, os testes e a manutenção.

\section{Levantamento de Requisitos}
\label{sec-fundteo-requisitos}

O levantamento de requisitos é uma das etapas mais críticas do desenvolvimento de software, pois envolve a identificação e documentação das necessidades e expectativas dos usuários e demais stakeholders~\cite{sommerville2011software}. Os requisitos podem ser classificados em funcionais, que descrevem as funcionalidades do sistema, e não funcionais, que tratam de restrições e qualidades, como desempenho, segurança e usabilidade. Uma boa especificação de requisitos é fundamental para o sucesso do projeto, evitando retrabalho e garantindo que o produto final atenda aos objetivos propostos. No contexto do Sistema Marvin, a especificação de requisitos foi detalhada em documento próprio~\cite{requisitos_marvin_core}, servindo de base para o desenvolvimento das funcionalidades e interfaces.

\section{Projeto de Software}
\label{sec-fundteo-projeto}

O projeto de software consiste na definição da arquitetura, dos componentes e das interfaces do sistema, a partir dos requisitos levantados. Essa etapa busca transformar as necessidades identificadas em uma solução técnica viável, considerando aspectos como modularidade, reutilização, escalabilidade e manutenibilidade~\cite{pressman2016engenharia}. No contexto de sistemas web, o projeto também envolve decisões sobre a experiência do usuário (UX), a navegação, a disposição dos elementos na interface e a integração entre frontend e backend. O projeto do Sistema Marvin foi elaborado considerando as melhores práticas de engenharia de software, com ênfase na modularidade e escalabilidade. O documento de projeto descreve a arquitetura, os componentes principais e as integrações do sistema~\cite{projeto_marvin_core}.

\section{Engenharia Web}
\label{sec-fundteo-engweb}

A Engenharia Web é um ramo específico da engenharia de software voltado para o desenvolvimento de aplicações baseadas na web. Ela abrange técnicas, metodologias e ferramentas para lidar com os desafios próprios desse ambiente, como a heterogeneidade de plataformas, a necessidade de interfaces responsivas, a segurança e a escalabilidade~\cite{gellersen2002web}. O desenvolvimento web moderno valoriza a experiência do usuário, a acessibilidade e a personalização das interfaces, aspectos essenciais para sistemas utilizados por diferentes perfis de usuários, como no caso do Sistema Marvin.

\section{Tecnologias Utilizadas}
\label{sec-fundteo-tecnologias}

O desenvolvimento do Sistema Marvin faz uso de tecnologias consolidadas e amplamente adotadas no contexto de aplicações web corporativas. Entre as principais, destacam-se:

\begin{itemize}
    \item \textbf{Jakarta EE}: Plataforma para desenvolvimento de aplicações corporativas em Java, utilizada como base do backend do sistema.
    \item \textbf{EJB (Enterprise JavaBeans)}: Componente da plataforma Jakarta EE para implementação da lógica de negócio.
    \item \textbf{JPA (Jakarta Persistence API)}: Responsável pelo mapeamento objeto-relacional e persistência dos dados.
    \item \textbf{CDI (Contexts and Dependency Injection)}: Gerenciamento de dependências e ciclo de vida dos componentes.
    \item \textbf{JSF (Jakarta Server Faces) e Facelets}: Frameworks para construção das interfaces web do sistema, permitindo a criação de páginas dinâmicas e reutilizáveis.
    \item \textbf{Jakarta Security}: Gerenciamento de autenticação e autorização dos usuários.
    \item \textbf{PostgreSQL}: Sistema gerenciador de banco de dados relacional utilizado para armazenamento das informações.
    \item \textbf{WildFly}: Servidor de aplicações utilizado para execução e gerenciamento do sistema.
    \item \textbf{IntelliJ IDEA}: Ambiente de desenvolvimento integrado (IDE) utilizado pela equipe para implementação e manutenção do código-fonte.
    \item \textbf{Git}: Ferramenta de controle de versão utilizada para colaboração e gerenciamento do histórico de desenvolvimento.
\end{itemize}

Essas tecnologias foram escolhidas por sua robustez, comunidade ativa e aderência às melhores práticas de desenvolvimento web, além de já serem utilizadas no contexto do Sistema Marvin, facilitando a integração das novas interfaces ao sistema existente.


\section{Trabalhos Relacionados}

Diversos estudos destacam a importância da personalização de interfaces em sistemas acadêmicos e administrativos~\cite{albert1999internet, souza-mylopoulos:spe13}. A literatura aponta que a adaptação das interfaces conforme o contexto de uso contribui para a eficiência e satisfação dos usuários.

\section{Considerações Finais}

A fundamentação teórica apresentada embasa as decisões de projeto e implementação das interfaces personalizadas no Sistema Marvin, destacando a importância da usabilidade, da personalização e do uso de tecnologias adequadas para o desenvolvimento web.



%%% Início de seção. %%%
\section{Seções e Subseções}
\label{sec-fundteo-secoes}

O documento é organizado em capítulos (\texttt{\textbackslash chapter\{\}}), seções (\texttt{\textbackslash section\{\}}), subseções (\texttt{\textbackslash subsection\{\}}), sub-subseções (\texttt{\textbackslash subsubsection\{\}}) e assim por diante. Atenção, porém, a não criar estruturas muito profundas (sub-sub-sub-...) pois o documento não fica bem estruturado.


%%% Início de seção. %%%
\subsection{Referências a seções}
\label{sec-fundteo-secoes-refs}

Cada parte do documento (capítulo, seção, etc.) deve possuir um rótulo logo abaixo de sua definição. Por exemplo, este capítulo é definido com \texttt{\textbackslash chapter\{Introdução\}} seguido por \texttt{\textbackslash label\{sec-fundteo\}}. Assim, podemos fazer referências cruzadas usando o comando \texttt{\textbackslash ref\{rótulo\}}: ``O Capítulo~\ref{sec-fundteo} começa com a Seção~\ref{sec-fundteo-secoes}, que é ainda subdividida nas subseções~\ref{sec-fundteo-secoes-refs} e~\ref{sec-fundteo-secoes-sobrerefs}.

Para melhor organização das partes do documento, sugere-se primeiro utilizar o prefixo \texttt{sec-} (para diferenciar de referências à figuras, tabelas, etc. quando usarmos o comando \texttt{\textbackslash ref\{\}}) e também representar a hierarquia das seções nos rótulos. Por exemplo, o Capítulo~\ref{sec-fundteo} tem rótulo \texttt{sec-fundteo}, sua Seção~\ref{sec-fundteo-secoes} tem rótulo \texttt{sec-fundteo-secoes} e a Subseção~\ref{sec-fundteo-secoes-refs} tem rótulo \texttt{sec-fundteo-secoes-refs}.



%%% Início de seção. %%%
\subsection{Sobre referências cruzadas}
\label{sec-fundteo-secoes-sobrerefs}

Nas próximas seções, veremos que é possível fazer referência cruzada não só a seções mas também a listagens de código, figuras, tabelas, etc. Em todos estes casos, quando nos referimos à Seção X, Listagem Y ou Figura Z, consideramos que estes são os nomes próprios destes elementos e, portanto, usa-se a primeira letra maiúscula. Isso pode ser visto na Subseção~\ref{sec-fundteo-secoes-refs}, acima. A exceção é quando nos referimos a vários elementos ao mesmo tempo, por exemplo: ``as subseções~\ref{sec-fundteo-secoes-refs} e~\ref{sec-fundteo-secoes-sobrerefs}''.

Por fim, ao usar o comando \texttt{\textbackslash ref\{\}}, sugere-se separá-lo da palavra que vem antes dele com um \textasciitilde\ ao invés de espaço. Por exemplo: \texttt{o Capítulo\textasciitilde \textbackslash ref\{sec-fundteo\}}. Isso faz com que o \latex não quebre linha entre a palavra \texttt{capítulo} e o número do capítulo.




%%% Início de seção. %%%
\section{Citações Bibliográficas}
\label{sec-fundteo-citacoes}

Este documento utiliza a ferramenta de gerenciamento de referências bibliográficas do \latex, chamada \emph{BibTeX}. O arquivo \texttt{bibliografia.bib}, referenciado no arquivo \latex principal deste documento, contém algumas referências bibliográficas de exemplo. Assim como capítulos, seções, etc., tais referências também possuem rótulos, especificados como primeiro parâmetro de cada entrada (ex.: \texttt{@incollection\{albert1999internet, ...\}}.

Sugere-se um padrão para rótulos de referências bibliográficas para que fique claro também no código \latex qual referência está sendo citada. Por exemplo, ao citar a referência \texttt{dijkstra1959note}, sabemos que é um artigo escrito por \emph{Souza} e outros, publicado no \emph{SESAS} em \emph{2013} (geralmente a pessoa que citou sabe que publicação é SESAS e quem é Souza).

Para citar uma referência bibliográfica contida no arquivo \emph{BibTeX}, basta usar seu rótulo como parâmetro de um de dois comandos possíveis de citação:

\begin{itemize}
\item O comando \texttt{\textbackslash cite\{\}} efetua uma citação tradicional, colocando o nome do(s) autor(es) e o ano entre parênteses. Por exemplo, \texttt{\textbackslash cite\{albert1999internet\}} é transformado em \cite{albert1999internet};





\item O comando \texttt{\textbackslash citeonline\{\}} efetua uma citação integrada ao texto, colocando o nome do(s) autor(es) direto no texto e somente o ano entre parênteses. Por exemplo, ``de acordo com \texttt{\textbackslash citeonline\{albert1999internet\}}'' é transformado em: de acordo com \citeonline{albert1999internet};
\end{itemize}

Também é possível citar vários trabalhos de uma só vez, separando os rótulos das referências bibliográficas com uma vírgula dentro do comando apropriado. Por exemplo, \texttt{\textbackslash cite\{dijkstra1959note,lattes1947observations\}} \cite{dijkstra1959note,lattes1947observations}.

Os trabalhos citados são automaticamente incluídos na seção de referências bibliográficas, ao final do documento. Tudo é formatado automaticamente segundo padrões da ABNT.



%%% Início de seção. %%%
\section{Listagens de Código}
\label{sec-fundteo-listagens}

O pacote \texttt{listings}, incluído neste template, permite a inclusão de listagens de código. Análogo ao já feito anteriormente, listagens possuem rótulos para que possam ser referenciadas e sugerimos uma regra de nomenclatura para tais rótulos: usar como prefixo o rótulo do capítulo, substituindo \texttt{sec-} por \texttt{lst-}.

A Listagem~\ref{lst-intro-exemplo}, por exemplo, possui o rótulo \texttt{lst-intro-exemplo} e representa o código que foi usado no próprio documento para exibir as listagens desta seção. Como podemos ver, a sugestão é que os arquivos de código sejam colocados dentro da pasta \texttt{codigos/} e tenham nome idêntico ao rótulo, colocando a extensão adequada ao tipo de código.

\lstinputlisting[label=lst-intro-exemplo, caption=Exemplo de código \latex para inclusão de listagens de código., float=htpb]{codigos/lst-intro-exemplo.tex}

A Listagem~\ref{lst-intro-outroexemplo} mostra um exemplo de listagem com especificação da linguagem utilizada no código. O pacote \texttt{listings} reconhece algumas linguagens\footnote{Veja a lista de linguagens suportadas em \url{http://en.wikibooks.org/wiki/LaTeX/Source\_Code\_Listings\#Supported_languages}.} e faz ``coloração'' de código (na verdade, usa \textbf{negrito} e não cores) de acordo com a linguagem. O parâmetro \texttt{float=htpb} incluído em ambos os exemplos impede que a listagem seja quebrada em diferentes páginas.

\lstinputlisting[label=lst-intro-outroexemplo, caption=Exemplo de código \java especificando linguagem utilizada., language=Java, float=htpb]{codigos/lst-intro-outroexemplo.java}



%%% Início de seção. %%%
\section{Figuras}
\label{sec-fundteo-figuras}

Figuras podem ser inseridas no documento usando o \emph{ambiente} \texttt{figure} (ou seja, \texttt{\textbackslash begin\{figure\}} e \texttt{\textbackslash end\{figure\}}) e o comando \texttt{\textbackslash includegraphics\{\}}. Existem alguns outros elementos e propriedades úteis de serem configuradas, resultando no código exibido na Listagem~\ref{lst-intro-figuras}.

\lstinputlisting[label=lst-intro-figuras, caption=Código \latex utilizado para inclusão das figuras na Seção~\ref{sec-fundteo-figuras}., float=htpb]{codigos/lst-intro-figuras.tex}

O comando \texttt{\textbackslash centering} centraliza a figura na página. A opção \texttt{width} do comando \texttt{\textbackslash includegraphics\{\}} determina o tamanho da figura e usa-se \texttt{\textbackslash textwidth} (opcionalmente multiplicado por um número) para se referir à largura da página.

O parâmetro do comando \texttt{\textbackslash includegraphics\{\}} indica onde a imagem pode ser encontrada. Foi criado o diretório \texttt{figuras/} para conter as figuras do documento, dando uma melhor organização aos arquivos. Ao abrir esta pasta, repare que as figuras possuem duas versões---uma em \texttt{.eps} e outra em \texttt{.pdf}---e que o comando \texttt{\textbackslash includegraphics\{\}} não especifica a extensão. Isso se dá porque o \latex possui um compilador para formato PostScript (\texttt{latex}) que espera as imagens em \texttt{.eps} e um compilador para PDF (\texttt{pdflatex}) que espera as imagens em \texttt{.pdf}.

Por fim, o comando \texttt{\textbackslash caption\{\}} especifica a descrição da figura e \texttt{\textbackslash label\{\}}, como de costume, estabelece um rótulo para permitir referência cruzada de figuras. Note ainda que é utilizada a mesma estratégia de nomenclatura de rótulos usada nas listagens, porém utilizando o prefixo \texttt{fig-}.

As figuras~\ref{fig-fundteo-exemplo} e~\ref{fig-fundteo-exemplosideways} mostram o resultado do código da Listagem~\ref{lst-intro-figuras}. A Figura~\ref{fig-fundteo-exemplosideways}, em particular, utiliza o pacote \texttt{rotating} para mostrar figuras largas em modo paisagem. Basta usar o ambiente \texttt{sidewaysfigure} ao invés de \texttt{figure}.

\begin{figure}
\centering
\includegraphics[width=.25\textwidth]{figuras/fig-fundteo-exemplo.png} 
\caption{Exemplo de figura.}
\label{fig-fundteo-exemplo}
\end{figure}

\begin{sidewaysfigure}
\centering
\includegraphics[width=\textwidth]{figuras/fig-fundteo-exemplosideways} 
\caption{Exemplo de figura em modo paisagem: um modelo de objetivos~\cite{souza-mylopoulos:spe13}.}
\label{fig-fundteo-exemplosideways}
\end{sidewaysfigure}



%%% Início de seção. %%%
\section{Tabelas}
\label{sec-fundteo-tabelas}

Tabelas são um ponto fraco do \latex. Elas são complicadas de fazer e, dependendo da complexidade da tabela (muitas células mescladas, por exemplo), vale a pena construi-las em outro programa (por exemplo, em seu editor de texto favorito), converter para PDF e inclui-las no documento como figuras. Mostramos, no entanto, alguns exemplos de tabela a seguir. O código utilizado para criar as tabelas encontra-se nas listagens~\ref{lst-intro-tabelas01} e~\ref{lst-intro-tabelas02}. 

\lstinputlisting[label=lst-intro-tabelas01, caption=Código \latex utilizado para inclusão das tabelas~\ref{tbl-intro-exemplo01} e~\ref{tbl-intro-exemplo02}., float=htpb]{codigos/lst-intro-tabelas01.tex}

\lstinputlisting[label=lst-intro-tabelas02, caption=Código \latex utilizado para inclusão da Tabela~\ref{tbl-intro-exemplo03}., float=htpb]{codigos/lst-intro-tabelas02.tex}

% Exemplo de tabela 01:
\begin{table}
\caption{Exemplo de tabela com diferentes alinhamentos de conteúdo.}
\label{tbl-intro-exemplo01}
\centering
\begin{tabular}{ | c | l | r | p{40mm} |}\hline
\textbf{Centralizado} & \textbf{Esquerda} & \textbf{Direita} & \textbf{Parágrafo}\\\hline
C & L & R & Alinhamento de tipo parágrafo especifica largura da coluna e quebra o texto automaticamente.\\
\hline
Linha 2 & Linha 2 & Linha 2 & Linha 2\\
\hline
\end{tabular}
\end{table}

% Exemplo de tabela 02:
\begin{table}
\caption{Exemplo que especifica largura de coluna e usa lista enumerada (adaptada de~\cite{souza-mylopoulos:spe13}).}
\label{tbl-intro-exemplo02}
\centering
\renewcommand{\arraystretch}{1.2}
\begin{small}
\begin{tabular}{ | p{15mm} | p{77mm} | p{55mm} |}\hline
\textbf{\textit{AwReq}} & \textbf{Adaptation strategies} & \textbf{Applicability conditions}\\\hline

AR1 &
\vspace{-2mm}\begin{enumerate}[topsep=0cm, partopsep=0cm, itemsep=0cm, parsep=0cm, leftmargin=0.5cm]
\item \textit{Warning(``AS Management'')}
\item \textit{Reconfigure($\varnothing$)}
\end{enumerate}\vspace{-4mm} &
\vspace{-2mm}\begin{enumerate}[topsep=0cm, partopsep=0cm, itemsep=0cm, parsep=0cm, leftmargin=0.5cm]
\item Once per adaptation session;
\item Always.
\end{enumerate}\vspace{-4mm}
\\\hline

AR2 &
\vspace{-2mm}\begin{enumerate}[topsep=0cm, partopsep=0cm, itemsep=0cm, parsep=0cm, leftmargin=0.5cm]
\item \textit{Warning(``AS Management'')}
\item \textit{Reconfigure($\varnothing$)}
\end{enumerate}\vspace{-4mm} &
\vspace{-2mm}\begin{enumerate}[topsep=0cm, partopsep=0cm, itemsep=0cm, parsep=0cm, leftmargin=0.5cm]
\item Once per adaptation session;
\item Always.
\end{enumerate}\vspace{-4mm}
\\\hline
\end{tabular}
\end{small}
\end{table}

% Exemplo de tabela 03:
\begin{table}
\caption{Exemplo que mostra equações em duas colunas (adaptada de~\cite{souza-mylopoulos:spe13}).}
\label{tbl-intro-exemplo03}
\centering
\vspace{1mm}
\fbox{\begin{minipage}{.98\linewidth}
\begin{minipage}{0.51\linewidth}
\vspace{-4mm}
\begin{eqnarray}
\Delta \left( I_{AR1} / NoSM \right) \left[ 0, maxSM \right] > 0\\
\Delta \left( I_{AR2} / NoSM \right) \left[ 0, maxSM \right] > 0\\
\Delta \left( I_{AR3} / LoA \right) < 0\\
\end{eqnarray}
\vspace{-6mm}
\end{minipage}
\hspace{2mm}
\vline 
\begin{minipage}{0.41\linewidth}
\vspace{-4mm}
\begin{eqnarray}
\Delta \left( I_{AR11} / VP2 \right) < 0\\
\Delta \left( I_{AR12} / VP2 \right) > 0\\
\Delta \left( I_{AR6} / VP3 \right) > 0\\
\end{eqnarray}
\vspace{-6mm}
\end{minipage}
\end{minipage}}
\end{table}

